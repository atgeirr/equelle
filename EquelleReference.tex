\documentclass[11pt]{article}
%
% definitions
\usepackage{amsmath,amssymb}
\usepackage{xcolor}
\usepackage{bm}
\usepackage{todonotes}
\setlength{\marginparwidth}{0.6in} % Needed for todonotes because geometry package is used
\usepackage[vlined,boxed]{algorithm2e}    % For those with brand new LaTeX

\newcommand{\cads}{a}
\newcommand{\eff}{\mathrm{eff}}
\newcommand{\cmax}{c_{\max}}
\newcommand{\Sb}{{\bm{S}}}
\newcommand{\cb}{{\bm{c}}}
\newcommand{\Kb}{{\bm{K}}}
\newcommand{\kb}{{\bm{k}}}
\newcommand{\nb}{{\bm{n}}}
\newcommand{\ub}{{\bm{u}}}
\newcommand{\abs}[1]{\left| #1\right|}
\newcommand{\norm}[1]{\left\| #1\right\|}
\newcommand{\dd}{\mathrm{d}}
\newcommand{\Sdpv}{\ensuremath{S_\text{dpv}}}
\newcommand{\fracpar}[2]{\frac{\partial #1}{\partial #2}}
\newcommand{\code}[1]{\texttt{#1}}

%% Extra markup for revisions:
%\newcommand{\rev}[1]{\textcolor{BrickRed}{#1}}

\title{Equelle Reference Manual}
\author{Atgeirr Fl{\o} Rasmussen, SINTEF ICT}

\begin{document}

\maketitle


%%%%%%%%%%%%%%%%%%%%%%%%%%%%%%%%%%%%%%%%%%%%%%%%%%%%%%%%%%%%%%%%%%%%%%%%%%%%%%
\section{Introduction}

Equelle is a domain-specific language intended for writing simulators, in the sense of
programs that numerically solve partial differential equations. The language was created
at SINTEF ICT, department of Applied Mathematics, starting in 2013.

\subsection{Motivation}

The primary motivation for Equelle is to separate the concerns of the application expert
who knows a field, its physics and equations from the computational scientist who knows
parallel programming and linear solvers.

We also seek to minimize the work needed to port simulators to new architectures and
hardware types. This is achieved by allowing multiple back-ends to the language, reducing
the question of supporting a new hardware type to implementing a new back-end.

We have chosen not to start at the level of the PDEs themselves. Instead, our intention is
that Equelle should make it easy to express the discrete equations for a simulator.
Furthermore, we have concentrated on features useful for finite volume methods, as such
methods are our primary focus.

\subsection{Vectorized notation}

In languages like C or Fortran, numerical programs usually have a large number of
loops. In Equelle we seek to have fewer loops, and especially to avoid indexing. Indexing
is a major source of errors in numerical computation, and we have sought to minimize them
by using vectorized notation. In Equelle, the statement \code{a = b + c*d} can just as
well refer to collections of number as to single numbers. All such statements are
interpreted element-wise, similar to Matlab's initial-point operators such as
\code{.*} or \code{./} which do not perform matrix operations but element-wise
operations.

\subsection{The grid concept in Equelle}

In any Equelle program, a computational grid is assumed to exist. It is constant and
unchanging from the view of the program, and it is up to the back-end to actually
create or read a particular grid, when the simulator is run.

Grids are cell complexes. They consist of entities of varying dimensionality. The entities
of highest dimension are called cells and all other entities can be obtained as
intersections of cells. The converse also holds: all possible intersections of cells are
entities of the grid. This means that Equelle grids are unstructured, and capable of
representing almost arbitrary computational grids. It also means that there is no notion
of non-conforming grid, since all cell-cell intersections are by definition entities of
the grid.

\subsection{The \code{Collection Of} types}

In most programming languages, a collection of data will be accessed either as a sequence,
or by random access, using indices. In Equelle, data will often be associated with a
subset of grid entities instead. A variable can be created that has one element per cell
in the grid, for example. The Equelle compiler ensures that such a variable cannot be used
in a context where it does not make sense, such as adding it to a variable containing one
element per face. An Equelle \code{Collection} always has a domain, and can be thought of
as a 1-1 mapping from that domain to the data. In traditional languages, such mappings
would implicitly be from the integer indices to the data. New domains can be created from
user input, to capture concepts such as applying a certain boundary condition to only part
of the boundary, or using different equations on part of the grid. There is also a set of
operators (\code{On}, \code{Extend}) that manipulates these mappings, giving a great deal
of flexibility without the risk of indexing errors.

\subsection{Type checking and type inference}

Equelle is strongly typed. Making a mistake in typing will usually cause a compile
error. This is a safety feature, and when combined with the domains of collections being
part of its type it is a strong safeguard against errors that are easy to make in other
languages.

Mostly, Equelle can be said to be statically typed. However, the domains of
\code{Collection}s can be dynamically computed or input, so that part of the type system
can be said to be dynamic. However, this is all checked at compile time.

\subsection{Declaration and assignment, immutability}

Unlike languages like C++ (at least before \code{auto} was allowed), the user is usually
not burdened with declaring the types of variables. The Equelle compiler is able to infer
the correct types of any valid expression. Therefore Equelle statements can be very
concise, yet unlike languages such as Python or Matlab, type safety is enforced at compile
time.

To declare a variable, use a single statement with a colon separating the variable name
and its type:

\code{volume : Scalar}

To assign a variable, use the = sign:

\code{volume = 5.67}

This can be combined, as follows:

\code{volume : Scalar = 5.67}

The declarations are entirely optional. They can be included as a form of documentation,
or left out according to the wishes of the programmer.

By default, Equelle variables are immutable, and can only be assigned to once. This makes
it easier for the compiler and back-end to parallelize programs correctly. It is possible
to create mutable variables, but it must be done explicitly, but declaring the variable to
be \code{Mutable}.



%%%%%%%%%%%%%%%%%%%%%%%%%%%%%%%%%%%%%%%%%%%%%%%%%%%%%%%%%%%%%%%%%%%%%%%%%%%%%%
\section{Types in Equelle}

\subsection{Basic types}

There are 7 basic types in Equelle, as shown in Table \ref{tab:basic-types}. Note that
there is no separate type for integer numbers. This helps avoiding some types of errors
that are widespread in other languages, for example the expression \code{1/2} will
yield an expression of type \code{Scalar} with the value 0.5, and not zero as in C.

\begin{table}
\begin{tabular}{l|l}
{\em Type} & {\em Semantics} \\
\hline \\
\code{Scalar} & A single floating-point number \\
\code{Vector} & A pair or triple of \code{Scalar}s (depending on grid dimension) \\
\code{Bool} & A flag that is \code{True} or \code{False} \\
\code{Cell} & A cell entity of the grid \\
\code{Face} & A face entity of the grid \\
\code{Edge} & An edge entity of the grid \\
\code{Vertex} & A vertex entity of the grid \\
\end{tabular}
\caption{Basic types in Equelle}
\label{tab:basic-types}
\end{table}

\subsection{Collections}

Variable can contain several elements of the same basic type. Such a variable is called a
collection.

\subsection{Domains and the \code{On} concept}

\subsection{The \code{Subset Of} concept}

\subsection{Sequences}

\subsection{Arrays}

\subsection{Function types}

\subsection{Mutable variables}


%%%%%%%%%%%%%%%%%%%%%%%%%%%%%%%%%%%%%%%%%%%%%%%%%%%%%%%%%%%%%%%%%%%%%%%%%%%%%%
\section{Built-in operators and functions}

\subsection{Arithmetic operators}

\subsection{Grid topology}

\subsection{Grid geometry}

\subsection{Discrete operators}

\subsection{Solver functions}

\subsection{Miscellaneous functions}

%%%%%%%%%%%%%%%%%%%%%%%%%%%%%%%%%%%%%%%%%%%%%%%%%%%%%%%%%%%%%%%%%%%%%%%%%%%%%%
\section{Control structures and functions}

\subsection{Scopes}

\subsection{The For construct}

\subsection{Functions}


%%%%%%%%%%%%%%%%%%%%%%%%%%%%%%%%%%%%%%%%%%%%%%%%%%%%%%%%%%%%%%%%%%%%%%%%%%%%%%
\section{The grammar of Equelle}

\subsection{Allowed identifiers}

\subsection{Syntax for literals}

\subsection{Expression syntax}

\subsection{Statements}

\subsection{The formal grammar definition}


%%%%%%%%%%%%%%%%%%%%%%%%%%%%%%%%%%%%%%%%%%%%%%%%%%%%%%%%%%%%%%%%%%%%%%%%%%%%%%
% Bibliography.

%\begin{small}
%  \bibliographystyle{spelike} %\bibliography{bib/polymer,bib/diagnostics,bib/co2,bib/fronttracking,bib/divSINTEF,bib/reorder,bib/streamlines,bib/misc}
%\end{small}


\end{document}


%% Local Variables:
%% fill-column: 90
%% mode: LaTeX
%% TeX-master: t
%% End:
